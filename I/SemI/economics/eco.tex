\documentclass[10pt]{article}
\usepackage[a4paper, total={6.5in, 9.5in}]{geometry}
\usepackage{datetime}

\newdate{date}{3}{10}{2017}

\title{Ecological Imbalance \& its Effect on the Economy of a Nation}
\author{NVS Kiran, Nagubandi Teja, Piyush Pandey, Prakruti Singh,\\
 Ritvik Raj, Rochan Avlur, Saisree Pokala}
\date{\displaydate{date}}

\begin{document}
\maketitle

The general definition of ecology somewhat follows the following - the scientific study of the distribution, abundance, relations and interactions of organisms with the environment. However, this definition can be further broadened to include the study of plant and animal populations, plant and animal communities and ecosystems. Ecosystems here, describes the web or network of relations among organisms at different scales of organization. The discipline of ecology emerged from the natural sciences in the late 19th century. Ecology is not synonymous with environment, environmentalism, or environmental science.
\\
\\
There are many practical applications of ecology in conservation biology, wetland management, natural resource management (agriculture, forestry, fisheries), city planning (urban ecology), community health, economics, basic \& applied science and it provides a conceptual framework for understanding and researching human social interaction (human ecology).
\\
\\
Economy on the other hand can be described as the management of resources that is accessible to the system. In most cases, the system refers to a country, thus increasing the complexity and volume of its resources. Naturally, one seems to have been defined particularly to make use of the other. Majority of the countries today continue to heavily rely on agriculture, hunting and fishing industries. These industries depend on the ecological balance of the respective organisms in the ecosystem.
\\
\\
As we inch closer every day into the second decade of the 21st century, technological advancements have exponentially made our lives easier by automating tasks that were previously performed by man. Historically, the stable ecological balance of a location could be linked to the lower rate of resource extraction. However, this did not remain the same as with the entrance of complex machinate and cheap automation technologies, the increase in the rate was exponential. This has lead to the ecological imbalances that we face today. Some of the side effects of these imbalances have been detailed below. 
\\
\\
Severe coral bleaching on Australia’s Great Barrier Reef may result in a loss of around \$750 million for the economy and tourism industry in the state of Queensland, a media report said on Wednesday. A report released by Australian Climate Council, ‘Climate Change: A Deadly Threat to Coral Reefs,’ looks at the economic impact of massive coral bleaching on the Great Barrier Reef in 2016 and 2017, the Efe news agency reported. About 1,500 km, or two-thirds, of this coral system in north-eastern Australiahas already been extremely damaged. The bleaching was caused by a rise in water temperatures due to climate change and was exacerbated by the effects of El Nino cycle in 2016. The data in the report has indicated that if severe bleaching on the Great Barrier Reef continues, this could result in the loss of more than one million visitors to the region annually, as well as 10,000 jobs.
\\
\\
The question to be asked is - How can a country regulate the rate at which it extracts resources while not greatly impacting its economy?

\end{document}